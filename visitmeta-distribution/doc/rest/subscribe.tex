\section{Subscribe Management}
The following section shows the handling of subscriptions. \{Subscription Name\} like \{Connection Name\} is a unique identifier which can be chosen freely.
\subsection{Get Subscriptions}
\begin{lstlisting}
HTTP:GET
http://example.com:8000/{Connection Name}/subscribe
\end{lstlisting}

\begin{minipage}{\linewidth}
\textbf{Example Request:}
\begin{lstlisting}
HTTP:GET
http://example.com:8000/default/subscribe
\end{lstlisting}
\end{minipage}

\begin{minipage}{\linewidth}
\textbf{Response:}
\begin{lstlisting}
["default", "exampleSub"]
\end{lstlisting}
\end{minipage}

If the suffix \textit{?onlyActive=true} is given, only active subscriptions will be returned.

\begin{lstlisting}
HTTP:GET
http://example.com:8000/default/subscribe?onlyActive=true
\end{lstlisting}


\subsection{Subscribe}
\begin{lstlisting}
HTTP:PUT
http://example.com:8000/{Connection Name}/subscribe/update
Content-Type: application/json
{
  "{Subscription Name}":
  {
    "startIdentifier": "{Identifier Name}",
    "identifierType": "{Identifier Type}"
  }
}
\end{lstlisting}
Identifier Types are:
\begin{itemize}
\item access-request
\item device
\item ip-address
\item mac-address
\end{itemize}

\begin{minipage}{\linewidth}
\textbf{Example Request:}
\begin{lstlisting}
HTTP:PUT
http://example.com:8000/default/subscribe/update
Content-Type: application/json
{
  "subExample":
  {
    "startIdentifier": "freeradius-pdp",
    "identifierType": "device"
  }
}
\end{lstlisting}
\end{minipage}

\begin{minipage}{\linewidth}
\textbf{Response:}
\begin{lstlisting}
INFO: subscribe successfully
\end{lstlisting}
\end{minipage}

Only to save subscriptions then use:

\begin{lstlisting}
HTTP:PUT
http://example.com:8000/default/subscribe
Content-Type: application/json
...
\end{lstlisting}

\subsection{Start Subscription}
\begin{lstlisting}
HTTP:PUT
http://example.com:8000/{Connection Name}/subscribe
	/start/{Subscription Name}
\end{lstlisting}

\textbf{Example Request:}
\begin{lstlisting}
HTTP:PUT
http://example.com:8000/default/subscribe/start/subExample
\end{lstlisting}

\textbf{Response:}
\begin{lstlisting}
INFO: subscription('subExample') enabled
\end{lstlisting}

\subsection{Stop Subscription}
\begin{lstlisting}
HTTP:PUT
http://example.com:8000/{Connection Name}/subscribe
	/stop/{Subscription Name}
\end{lstlisting}

\textbf{Example Request:}
\begin{lstlisting}
HTTP:PUT
http://example.com:8000/default/subscribe/stop/subExample
\end{lstlisting}

\textbf{Response:}
\begin{lstlisting}
INFO: subscription('subExample') disabled
\end{lstlisting}

\subsection{Delete Subscription}
\begin{lstlisting}
HTTP:DELETE
http://example.com:8000/{Connection Name}/
	subscribe/{Subscription Name}
\end{lstlisting}

\begin{minipage}{\linewidth}
\textbf{Example Request:}
\begin{lstlisting}
HTTP:DELETE
http://example.com:8000/default/subscribe/exampleSub
\end{lstlisting}
\end{minipage}

\begin{minipage}{\linewidth}
\textbf{Response:}
\begin{lstlisting}
INFO: delete subscription(exampleSub) successfully
\end{lstlisting}
\end{minipage}

\subsection{Delete All Subscriptions}
\begin{lstlisting}
HTTP:DELETE
http://example.com:8000/{Connection Name}/
	subscribe?deleteAll=true
\end{lstlisting}

\begin{minipage}{\linewidth}
\textbf{Example Request:}
\begin{lstlisting}
HTTP:DELETE
http://example.com:8000/default/subscribe?deleteAll=true
\end{lstlisting}
\end{minipage}

\begin{minipage}{\linewidth}
\textbf{Response:}
\begin{lstlisting}
INFO: delete all subscriptions successfully
\end{lstlisting}
\end{minipage}