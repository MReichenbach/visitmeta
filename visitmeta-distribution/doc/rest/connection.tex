\section{Connection Management}
This section shows examples using the REST-Interface to manage connections from the dataservice to any given number of map-server. The \{Connection Name\} is a unique name for each connection which can be chosen freely.

\subsection{Get Connections}
\textbf{Example Request:}
\begin{lstlisting}
HTTP:GET
http://example.com:8000
\end{lstlisting}

If the suffix \textit{?onlyActive=true} is given, only active connections will be returned.

\begin{lstlisting}
HTTP:GET
http://example.com:8000?onlyActive=true
\end{lstlisting}

\textbf{Response:}
\begin{lstlisting}
["default", "exampleConn"]
\end{lstlisting}
The Response returns a JSON-Array which contains every \{Connection Name\} saved in the dataservice. 

\subsection{Save Connection}
\begin{lstlisting}
HTTP:PUT
http://example.com:8000/
Content-Type: application/json
{
	connectionName:{Connection Name}
	ifmapServerUrl:{map-Server},
	userName:{Username},
	userPassword:{Password}
}
\end{lstlisting}
List of required parameters:
\begin{itemize}
\item connectionName
\item ifmapServerUrl
\item userName
\item userPassword
\end{itemize}
List of optional parameters:
\begin{itemize}
\item authenticationBasic
\item truststorePath
\item truststorePassword
\item useConnectionAsStartup
\item maxPollResultSize
\end{itemize}

\textbf{Example Request:}
\begin{lstlisting}
HTTP:PUT
http://example.com:8000/
Content-Type: application/json
{
	connectionName: exampleConn
	ifmapServerUrl:"https://localhost:8443",
	userName:visitmeta,
	userPassword:visitmeta
}
\end{lstlisting}

\textbf{Response:}
\begin{lstlisting}
exampleConn was saved
\end{lstlisting}

\subsection{Delete Connection}
\textbf{Not implemented as of \today}
\begin{lstlisting}
HTTP:DELETE
http://example.com:8000/{Connection Name}
\end{lstlisting}

\textbf{Example Request:}
\begin{lstlisting}
HTTP:DELETE
http://example.com:8000/default
\end{lstlisting}

\textbf{Response:}
\begin{lstlisting}
Not implemented
\end{lstlisting}

\subsection{Connect}
\begin{lstlisting}
HTTP:PUT
http://example.com:8000/{Connection Name}/connect
\end{lstlisting}

\textbf{Example Request:}
\begin{lstlisting}
HTTP:PUT
http://example.com:8000/default/connect
\end{lstlisting}

\textbf{Response:}
\begin{lstlisting}
INFO: connecting successfully
\end{lstlisting}

\subsection{Disconnect}
\begin{lstlisting}
HTTP:PUT
http://example.com:8000/{Connection Name}/disconnect
\end{lstlisting}

\textbf{Example Request:}
\begin{lstlisting}
HTTP:PUT
http://example.com:8000/default/disconnect
\end{lstlisting}

\textbf{Response:}
\begin{lstlisting}
INFO: disconnection successfully
\end{lstlisting}